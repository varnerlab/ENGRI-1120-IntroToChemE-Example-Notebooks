\begin{figure*}[h!]\centering
    \captionsetup{width=0.85\linewidth}
    \includegraphics[width=0.85\textwidth]{./figs/Fig-Pxy-acetone-water-85C-AP2.pdf}
    \caption{Pressure (kPa) versus composition (x$_{1}$ or y$_{1}$) for a binary mixture of Acetone(1)/Water(2) computed assuming ideal liquid and vapor phases.}\label{fig-VLE-ideal-problem}
    \end{figure*}
    
    \item{(25 points) Cornell Inc. was hired to design a flash separation process for a binary ($\mathcal{M}$ = 2) mixture of Acetone(1)/Water(2). 
    The engineering team has been asked to do initial design calculations using an ideal Pxy diagram (Fig. \ref{fig-VLE-ideal-problem}). 
    
    Let the saturation pressure of component $i$ be described by the Antoine equation:
    \begin{equation}
      \ln\left(P_{i}^{sat}\right) = A_{i} - \frac{B_{i}}{C_{i}+T}
    \end{equation}where $P_{i}^{sat}$ has units of kPa and the temperature $T$ has units of $^{\circ}C$.
    The Antoine parameters are given by:

    \begin{table}[!ht]
      \centering
      \caption{Antoine parameters for the Acetone/Water flash problem.}
      \setlength{\tabcolsep}{18pt}
      \begin{tabular}{c|c|c|c}\toprule
        Species & A & B & C \\ \toprule
        Acetone & 14.31 & 2756.22 & 228.06 \\
        Water & 16.39 & 3885.7 & 230.17 \\\bottomrule
      \end{tabular}
    \end{table}
         
    \textbf{Assumptions}: (i) the Flash drum operates at steady-state;
    (ii) vapor-liquid equilibrium occurs everywhere inside the drum at some (T,P);
    (iii) treat both the vapor and liquid phases as ideal;
    (iv) the Flash drum is well-mixed;
    (v) a single liquid feed (stream 1) enters, and a vapor (stream 2) and liquid (stream 3) exit the drum;
    (vi) R = 8.314$\times$10$^{-2}$ L bar K$^{-1}$ mol$^{-1}$.
    
    \begin{itemize}
      \item[a)]{(3 points)~What temperature T ($^{\circ}$C) is the Flash drum operating at? (place your estimated temperature in Table
      \ref{tbl-state-flash-problem}).}    
      \item[b)]{(20 points)~Graphically estimate the exit composition and compute the missing values in Table \ref{tbl-state-flash-problem} assuming the Flash drum operates at P = 150 kPa with an input feed rate of $\dot{F}$ = 10 mol/t and $z_{1}$ = 0.64.}
      \item[c)]{(2 points)~Check your graphical composition estimates by computing the residual $\epsilon$ from the pressure expression:
        \begin{equation}
          \epsilon \equiv x_{1}P_{1}^{sat}+x_{2}P_{2}^{sat} - P
        \end{equation}
      If $\text{abs}(\epsilon)>5\%$, then recompute your values for part b (and update your values in Table \ref{tbl-state-flash-problem}).}
    \end{itemize}
    
    \clearpage
    
    \begin{table}[!ht]
      \centering
      \caption{State table for the Pxy flash problem.}\label{tbl-state-flash-problem}
      \renewcommand{\arraystretch}{2.0}
      \setlength{\tabcolsep}{18pt}
      \begin{tabular}{c|c|c|c|c|c|c}\toprule
      Stream & State & T ($^{\circ}$C) & $\dot{n}_{s,T}$ (mol/t) & $x_{1}$ or $y_{1}$ & $x_{2}$ or $y_{2}$ & P (kPa) \\ \toprule
      1 & L & N/A & 10 & 0.64 & 0.36 & N/A \\ \hline
      2 & V & & & & &  \\ \hline
      3 & L & & & & &  \\ \bottomrule
      \end{tabular}
    \end{table}}